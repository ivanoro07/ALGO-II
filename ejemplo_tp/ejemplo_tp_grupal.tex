\documentclass[10pt,a4paper]{article}

\usepackage[spanish,activeacute,es-tabla]{babel}
\usepackage[utf8]{inputenc}
\usepackage{ifthen}
\usepackage{listings}
\usepackage{dsfont}
\usepackage{subcaption}
\usepackage{amsmath}
\usepackage[strict]{changepage}
\usepackage[top=1cm,bottom=2cm,left=1cm,right=1cm]{geometry}%
\usepackage{color}%
\newcommand{\tocarEspacios}{%
	\addtolength{\leftskip}{3em}%
	\setlength{\parindent}{0em}%
}

% Especificacion de procs

\newcommand{\In}{\textsf{in }}
\newcommand{\Out}{\textsf{out }}
\newcommand{\Inout}{\textsf{inout }}

\newcommand{\encabezadoDeProc}[4]{%
	% Ponemos la palabrita problema en tt
	%  \noindent%
	{\normalfont\bfseries\ttfamily proc}%
	% Ponemos el nombre del problema
	\ %
	{\normalfont\ttfamily #2}%
	\
	% Ponemos los parametros
	(#3)%
	\ifthenelse{\equal{#4}{}}{}{%
		% Por ultimo, va el tipo del resultado
		\ : #4}
}

\newenvironment{proc}[4][res]{%
	
	% El parametro 1 (opcional) es el nombre del resultado
	% El parametro 2 es el nombre del problema
	% El parametro 3 son los parametros
	% El parametro 4 es el tipo del resultado
	% Preambulo del ambiente problema
	% Tenemos que definir los comandos requiere, asegura, modifica y aux
	\newcommand{\requiere}[2][]{%
		{\normalfont\bfseries\ttfamily requiere}%
		\ifthenelse{\equal{##1}{}}{}{\ {\normalfont\ttfamily ##1} :}\ %
		\{\ensuremath{##2}\}%
		{\normalfont\bfseries\,\par}%
	}
	\newcommand{\asegura}[2][]{%
		{\normalfont\bfseries\ttfamily asegura}%
		\ifthenelse{\equal{##1}{}}{}{\ {\normalfont\ttfamily ##1} :}\
		\{\ensuremath{##2}\}%
		{\normalfont\bfseries\,\par}%
	}
	\renewcommand{\aux}[4]{%
		{\normalfont\bfseries\ttfamily aux\ }%
		{\normalfont\ttfamily ##1}%
		\ifthenelse{\equal{##2}{}}{}{\ (##2)}\ : ##3\, = \ensuremath{##4}%
		{\normalfont\bfseries\,;\par}%
	}
	\renewcommand{\pred}[3]{%
		{\normalfont\bfseries\ttfamily pred }%
		{\normalfont\ttfamily ##1}%
		\ifthenelse{\equal{##2}{}}{}{\ (##2) }%
		\{%
		\begin{adjustwidth}{+5em}{}
			\ensuremath{##3}
		\end{adjustwidth}
		\}%
		{\normalfont\bfseries\,\par}%
	}
	
	\newcommand{\res}{#1}
	\vspace{1ex}
	\noindent
	\encabezadoDeProc{#1}{#2}{#3}{#4}
	% Abrimos la llave
	\par%
	\tocarEspacios
}
{
	% Cerramos la llave
	\vspace{1ex}
}

\newcommand{\aux}[4]{%
	{\normalfont\bfseries\ttfamily\noindent aux\ }%
	{\normalfont\ttfamily #1}%
	\ifthenelse{\equal{#2}{}}{}{\ (#2)}\ : #3\, = \ensuremath{#4}%
	{\normalfont\bfseries\,;\par}%
}

\newcommand{\pred}[3]{%
	{\normalfont\bfseries\ttfamily\noindent pred }%
	{\normalfont\ttfamily #1}%
	\ifthenelse{\equal{#2}{}}{}{\ (#2) }%
	\{%
	\begin{adjustwidth}{+2em}{}
		\ensuremath{#3}
	\end{adjustwidth}
	\}%
	{\normalfont\bfseries\,\par}%
}

% Tipos

\newcommand{\nat}{\ensuremath{\mathds{N}}}
\newcommand{\ent}{\ensuremath{\mathds{Z}}}
\newcommand{\float}{\ensuremath{\mathds{R}}}
\newcommand{\bool}{\ensuremath{\mathsf{Bool}}}
\newcommand{\cha}{\ensuremath{\mathsf{Char}}}
\newcommand{\str}{\ensuremath{\mathsf{String}}}

% Logica

\newcommand{\True}{\ensuremath{\mathrm{true}}}
\newcommand{\False}{\ensuremath{\mathrm{false}}}
\newcommand{\Then}{\ensuremath{\rightarrow}}
\newcommand{\Iff}{\ensuremath{\leftrightarrow}}
\newcommand{\implica}{\ensuremath{\longrightarrow}}
\newcommand{\IfThenElse}[3]{\ensuremath{\mathsf{if}\ #1\ \mathsf{then}\ #2\ \mathsf{else}\ #3\ \mathsf{fi}}}
\newcommand{\yLuego}{\land _L}
\newcommand{\oLuego}{\lor _L}
\newcommand{\implicaLuego}{\implica _L}

\newcommand{\cuantificador}[5]{%
	\ensuremath{(#2 #3: #4)\ (%
		\ifthenelse{\equal{#1}{unalinea}}{
			#5
		}{
			$ % exiting math mode
			\begin{adjustwidth}{+2em}{}
				$#5$%
			\end{adjustwidth}%
			$ % entering math mode
		}
		)}
}

\newcommand{\existe}[4][]{%
	\cuantificador{#1}{\exists}{#2}{#3}{#4}
}
\newcommand{\paraTodo}[4][]{%
	\cuantificador{#1}{\forall}{#2}{#3}{#4}
}

%listas

\newcommand{\TLista}[1]{\ensuremath{seq \langle #1\rangle}}
\newcommand{\lvacia}{\ensuremath{[\ ]}}
\newcommand{\lv}{\ensuremath{[\ ]}}
\newcommand{\longitud}[1]{\ensuremath{|#1|}}
\newcommand{\cons}[1]{\ensuremath{\mathsf{addFirst}}(#1)}
\newcommand{\indice}[1]{\ensuremath{\mathsf{indice}}(#1)}
\newcommand{\conc}[1]{\ensuremath{\mathsf{concat}}(#1)}
\newcommand{\cab}[1]{\ensuremath{\mathsf{head}}(#1)}
\newcommand{\cola}[1]{\ensuremath{\mathsf{tail}}(#1)}
\newcommand{\sub}[1]{\ensuremath{\mathsf{subseq}}(#1)}
\newcommand{\en}[1]{\ensuremath{\mathsf{en}}(#1)}
\newcommand{\cuenta}[2]{\mathsf{cuenta}\ensuremath{(#1, #2)}}
\newcommand{\suma}[1]{\mathsf{suma}(#1)}
\newcommand{\twodots}{\ensuremath{\mathrm{..}}}
\newcommand{\masmas}{\ensuremath{++}}
\newcommand{\matriz}[1]{\TLista{\TLista{#1}}}
\newcommand{\seqchar}{\TLista{\cha}}

\renewcommand{\lstlistingname}{Código}
\lstset{% general command to set parameter(s)
	language=Java,
	morekeywords={endif, endwhile, skip},
	basewidth={0.47em,0.40em},
	columns=fixed, fontadjust, resetmargins, xrightmargin=5pt, xleftmargin=15pt,
	flexiblecolumns=false, tabsize=4, breaklines, breakatwhitespace=false, extendedchars=true,
	numbers=left, numberstyle=\tiny, stepnumber=1, numbersep=9pt,
	frame=l, framesep=3pt,
	captionpos=b,
}

\usepackage{caratula} % Version modificada para usar las macros de algo1 de ~> https://github.com/bcardiff/dc-tex


\titulo{Trabajo Práctico Nro 1: En busqueda del Camino}
\subtitulo{Especificación y WP}

\fecha{\today}

\materia{Algoritmos y Estructuras de Datos}
\grupo{punto\&\&coma}

\integrante{Oro, Iván}{667/24}{ivanoro0701@hotmail.com}
\integrante{Obregón, María}{706/24}{mobregonds@gmail.com}
\integrante{Alvarez Bertoya, Lautaro}{666/24}{lalvarezbertoya@gmail.com}
\integrante{Varela, Tomás}{438/24}{tlvarela2005@gmail.com }



\begin{document}

\maketitle{}
\section{Especificación}
\subsection*{Predicados de uso general}
\pred{tieneHabitantes}{ciudad:{Ciudad}}{ciudad.habitantes\geq 0}

\pred{noTieneRepetidos}{secuencia1:\TLista{Ciudad}}{\paraTodo [unalinea]{i}{\ent}{0 \leq i<|secuencia1| \implicaLuego \neg \existe [unalinea]{j}{\ent}{0 \leq j<|secuencia1| \land i\neq j \yLuego \\ secuencia1[i] = secuencia1[j]}}} 

\pred{tieneCamino}{distancias:\TLista{\TLista{\ent}}, desde:\ent , hasta:\ent  }
{\existe[unalinea]{lista}{\TLista{\ent}}{(|lista| > 1) \land \paraTodo[unalinea]{i}{\ent}{1\leq i< |lista| \implicaLuego ((lista[0]=desde)\land(lista[|lista|-1]=hasta) \land \\ \paraTodo[unalinea]{a}{\ent}{0\leq a<|lista| \implicaLuego 0 \leq lista[a] < |distancias|)\yLuego (distancias[lista[i]][lista[i-1]]>0) }}}}

\pred{esSimetrica}{matriz: \TLista{\TLista{\ent}}}{\paraTodo [unalinea]{i}{\ent}{0\leq i < |matriz| \implicaLuego \paraTodo [unalinea]{j}{\ent}{0\leq j<|matriz|\implicaLuego matriz[i][j] = matriz[j][i]}}}

\pred{esCuadrada}{matriz: \TLista{\TLista{\ent}}}{\paraTodo [unalinea]{i}{\ent}{0 \leq i<|matriz| \implicaLuego |matriz| = |matriz[i]|}}

\pred{diagonalEnCero}{matriz: \TLista{\TLista{\ent}}}{\paraTodo[unalinea]{i}{\ent}{0 \leq i < |matriz| \implicaLuego matriz[i][i]=0}}

\subsection {grandesCiudades}
A partir de una lista de ciudades, devuelve aquellas que tienen más de 50.000 habitantes.
\vspace{0.2cm}
\begin{proc}{grandesCiudades}{\In ciudades : \TLista{Ciudad} }{\TLista{Ciudad}}
	\requiere{\paraTodo[unalinea]{i}{\ent}{0\leq i < |ciudades| \implicaLuego tieneHabitantes(ciudades[i])}}
    \requiere{noTieneRepetidos(ciudades)}
	\asegura{\paraTodo [unalinea]{a}{\ent}{0 \leq a<|res| \implicaLuego (res[a]  \in  ciudades)}}
	\asegura{\paraTodo [unalinea]{j}{\ent}{0 \leq j<|ciudades| \implicaLuego (tieneMasDe50k(ciudades[j])  \land ciudades[j]  \in  res) \land \\(\neg tieneMasDe50k(ciudades[j])  \land ciudades[j]  \notin  res)}}
	\asegura{noTieneRepetidos(res)}
	
	\vspace{0.2cm}
	
\end{proc}
\pred{tieneMasDe50k}{ciudad:\TLista{Ciudad} }{ciudad.habitantes >50000} 


\subsection {sumaDeHabitantes}
Por cuestiones de planificación urbana, las ciudades registran sus habitantes mayores de edad
por un lado y menores de edad por el otro. Dadas dos listas de ciudades del mismo largo con los mismos nombres, una
con sus habitantes mayores y otra con sus habitantes menores, este procedimiento debe devolver una lista de ciudades
con la cantidad total de sus habitantes.
\vspace{0.3cm}
\begin{proc}{sumaDeHabitantes}{\In menoresDeCiudades : \TLista{Ciudad}, \In mayoresDeCiudades : \TLista{Ciudad} }{\TLista{Ciudad}}
	\requiere{mismoLargo(menoresDeCiudades, mayoresDeCiudades)}
	\requiere{noTieneRepetidos(menoresDeCiudades) \land  noTieneRepetidos(mayoresDeCiudades) }
	\requiere{\paraTodo[unalinea]{i}{\ent}{0\leq i < |menoresDeCiudades| \implicaLuego tieneHabitantes(menoresDeCiudades[i]) \land \\ 			tieneHabitantes(mayoresDeCiudades[i]) }}

	\requiere{\paraTodo [unalinea]{i}{\ent}{0 \leq i<|menoresDeCiudades| \implicaLuego \existe [unalinea]{j}{\ent}{0 \leq j<|mayoresDeCiudades| \yLuego 	\\		             menoresDeCiudades[i].nombre = mayoresDeCiudades[j].nombre}}}

	\asegura{mismoLargo(res, menoresDeCiudades) \land (\paraTodo [unalinea]{a}{\ent}{0 \leq a<|menoresDeCiudades| \implicaLuego \\ \existe [unalinea]{b}{\ent}{0 \leq b<|res| \yLuego 			             res[b].nombre = menoresDeCiudades[a].nombre)}}}
	\asegura{\paraTodo [unalinea]{m}{\ent}{0 \leq m<|menoresDeCiudades| \implicaLuego \existe [unalinea]{n}{\ent}{0 \leq n<|mayoresDeCiudades|   \ \yLuego 			             menoresDeCiudades[m].nombre = mayoresDeCiudades[n].nombre \ \land \\\ agruparHabitantes(menoresDeCiudades[m], mayoresDeCiudades[n]) \in res}}}
	\vspace{0.3cm}


\end{proc}
	\pred{mismoLargo}{secuencia1:\TLista{Ciudad}, secuencia2:\TLista{Ciudad} }{|secuencia1|=|secuencia2|} 

\vspace{0.2cm}
\aux{agruparHabitantes}{menoresDeCiudad: \ensuremath{Ciudad}, mayoresDeCiudad:\ensuremath{Ciudad}}{\ensuremath{Ciudad} \\}{\langle menoresDeCiudad.nombre, (menoresDeCiudad.habitantes+mayoresDeCiudad.habitantes)\rangle}
	\vspace{0.2cm}
\subsection {hayCamino}
 Un mapa de ciudades está conformada por ciudades y caminos que unen a algunas de ellas. A partir de
este mapa, podemos definir las distancias entre ciudades como una matriz donde cada celda i, j representa la distancia
entre la ciudad i y la ciudad j. Una distancia de 0 equivale a no haber camino directo entre i y j. Notar que
la distancia de una ciudad hacia sí misma es cero y la distancia entre A y B es la misma que entre B y A.
Dadas dos ciudades y una matriz de distancias, se pide determinar si existe un camino entre ambas ciudades.
\vspace{0.2cm}
\begin{proc}{hayCamino}{\In distancias: \TLista{\TLista{\ent}}, \In desde:\ent , \In hasta:\ent }{Bool}
	\requiere{esCuadrada(distancias)}
	\requiere{esSimetrica(distancias)}
	\requiere{diagonalEnCero(distancias)}
	\requiere{0\leq desde < |distancias| \land 0 \leq hasta < |distancias|}
	\requiere{\paraTodo [unalinea]{i}{\ent}{0 \leq i<|distancias| \implicaLuego \paraTodo [unalinea]{j}{\ent}{0 \leq j<|distancias[i]| \implicaLuego 	\\		             (distancias[i][j] \geq0)}}}
	\asegura{res = True \Iff tieneCamino(distancia, desde, hasta)}

	
\vspace{0.2cm}
	
\end{proc}
\vspace{0.2cm}

\subsection {cantidadCaminosNSaltos}
Dentro del contexto de redes informáticas, nos interesa contar la cantidad de “saltos”
que realizan los paquetes de datos, donde un salto se define como pasar por un nodo.
Así como definimos la matriz de distancias, podemos definir la matriz de conexión entre nodos, donde cada celda i, j
tiene un 1 si hay un único camino a un salto de distancia entre el nodo i y el nodo j, y un 0 en caso contrario. En este
caso, se trata de una matriz de conexión de orden 1, ya que indica cuáles pares de nodos poseen 1 camino entre ellos a
1 salto de distancia.
Dada la matriz de conexi´on de orden 1, este procedimineto debe obtener aquella de orden n que indica cu´antos caminos
de n saltos hay entre los distintos nodos. Notar que la multiplicación de una matriz de conexión de orden 1 consigo
misma nos da la matriz de conexión de orden 2, y así sucesivamente.
\vspace{0.3cm}
\begin{proc}{cantidadCaminosNSaltos}{\Inout conexion: \TLista{\TLista{\ent}}, \In n:\ent }{}
	\requiere{conexion=conexion_0}
	\requiere{n \geq 1}
	\requiere{esCuadrada(conexion)}
	\requiere{esSimetrica(conexion)}
	\requiere{diagonalEnCero(conexion)}
	\requiere{\paraTodo [unalinea]{i}{\ent}{0 \leq i<|conexion| \implicaLuego \paraTodo [unalinea]{j}{\ent}{0 \leq j<|conexion[i]| \implicaLuego 	\\		             (conexion[i][j] \in \{0,1\})}}}
	\asegura{\existe[unalinea]{listaMatrices}{\TLista{\TLista{\TLista{\ent}}}}{(|listaMatrices|=n)\land \paraTodo [unalinea]{q}{\ent}{1\leq q < |ListaMatrices| \implicaLuego     \\  (listaMatrices[0] = conexion_0) \land (|listaMatrices[q]| = |conexion_0|) \land esCuadrada(listaMatrices[q]) \\\land m3EsM2PorM1(listaMatrices[q],listaMatrices[q-1],listaMatrices[0]) \land conexion = listaMatrices[n-1]}}}


	\vspace{0.2cm}
	
\end{proc}

\pred{m3EsM2PorM1}{m3,m2,m1:\TLista{\TLista{\ent}}}{esSimetrica(m3) \land esSimetrica(m2) \land esSimetrica(m1) \land |m3|=|m2| \land |m2| = |m1|\yLuego \\{\paraTodo[unalinea]{a,b}{\ent}{0\leq a,b < |listaMatrices[0]| \implicaLuego   (m3[a][b] = \sum\limits_{k=0}^{|m3|-1} m2[a][k] * m1[k][b] )}}}

\subsection {caminoMínimo}
Dada una matriz de distancias, una ciudad de origen y una ciudad de destino, este procedimiento
debe devolver la lista de ciudades que conforman el camino con menor distancia entre ambas. En caso de no existir un
camino, se debe devolver una lista vacía.

\vspace{0.2cm}
\begin{proc}{caminoMínimo}{\In distancias: \TLista{\TLista{\ent}}, \In origen:\ent , \In destino:\ent }{\TLista{\TLista{\ent}}}
	\requiere{esCuadrada(distancias)}
	\requiere{esSimetrica(distancias)}
	\requiere{diagonalEnCero(conexion)}
	\requiere{\paraTodo [unalinea]{i}{\ent}{0 \leq i<|distancias| \implicaLuego \paraTodo [unalinea]{j}{\ent}{0 \leq j<|distancias[i]| \implicaLuego 	\\		             (distancias[i][j] \geq0)}}}
	\requiere{0\leq origen < |distancias| \land 0 \leq destino < |distancias|}

	\asegura{res = [] \Iff  \neg tieneCamino(distancias, origen, destino)}
	\asegura{\existe[unalinea]{lista}{\TLista{\ent}}{esCamino(distancias, lista, origen, destino)\land \\ \paraTodo [unalinea]{listaAux}{\TLista{\ent}} {esCamino(distancias, listaAux, origen, destino) \implicaLuego \\ \sum\limits_{k=1}^{|lista|-1} distancias[lista[k]][lista[k-1]] \leq 	\sum\limits_{k=1}^{|listaAux|-1} distancias[listaAux[k]][listaAux[k-1]]   }\land res = lista} }


	\vspace{0.2cm}
	
\end{proc}
\pred{esCamino}{distancias: \TLista{\TLista{\ent}}, lista:\TLista{\ent}, origen:\ent , destino: \ent}{tieneCamino(distancias, origen, destino) \land \paraTodo[unalinea]{j}{\ent}{ 
0\leq j<|lista| \implicaLuego lista[j] < |distancias|} \yLuego \\ \paraTodo [unalinea]{i}{\ent}{1\leq i<|lista| \implicaLuego distancias[lista[i]][lista[i-1]]>0}\land (origen = lista[0]) \land (destino = lista [|lista|-1])}


\section{Demostraciones de Correctitud}
La función poblaciónTotal recibe una lista de ciudades donde al menos una de ellas es grande (es decir, supera los
50.000 habitantes) y devuelve la cantidad total de habitantes. Dada la siguiente especificación:
\begin{proc}{poblacionTotal}{\In ciudades: \TLista{Ciudad}}{\ent}
	\requiere{\existe[unalinea]i\ent {0\leq i<|ciudades| \yLuego ciudades[i].habitantes > 50,000} \land \\
    \paraTodo[unalinea]i{\ent}{0 \leq i<|conexion|} \implicaLuego {(ciudades[i].habitantes \geq 0) } \land \\ 
    \paraTodo[unalinea]i{\ent}{\paraTodo[unalinea]j{\ent}{0 \leq i < j<ciudades[i].nombre \neq ciudades[j].nombre}}}
    \asegura{res = \sum\limits_{i=0}^{|ciudades|-1} ciudades[i].habitantes}
    
	\vspace{0.1cm}

\end{proc}
\begin{lstlisting}
res = 0;
i = 0;
while (i < ciudades.length) do
	res := res + ciudades[i].habitantes;
	i = i + 1
endwhile
	\end{lstlisting}
\vspace{0.3cm}
1. Demostrar que la implementación es correcta con respecto a la especificación \\
 \\
Lo probaremos por monotonía. Primero demostraremos la correctitud del ciclo del programa, es fundamental verificar tanto el Teorema del Invariante, que asegura que el invariante del ciclo se mantiene durante su ejecución, como el Teorema de Terminación, que garantiza que el ciclo finalizará. Estos teoremas son esenciales para determinar si la tripla de Hoare \{P\} S \{Q\} es correcta, donde \{P\} es la precondición, S el cuerpo del ciclo, y \{Q\} la postcondición. Solo si ambas propiedades se cumplen, podemos afirmar que el ciclo es correcto.\\ \\
{$
Pc = \{ res = 0 \land i = 0 \land (\existe[unalinea]i\ent {0\leq i<|ciudades| \yLuego ciudades[i].habitantes > 50,000} \land \\ \paraTodo[unalinea]i{\ent}{0 \leq i<|conexion| \implicaLuego |ciudades[i].habitantes \geq 0) } \land 
    \paraTodo[unalinea]i{\ent}{\paraTodo[unalinea]j{\ent}{0 \leq i < j<ciudades[i].nombre \neq ciudades[j].nombre}})\}\\
Qc = \{{res = \sum\limits_{i=0}^{|ciudades|-1}ciudades[i].habitantes}\}\\
B = \{{i< |ciudades|}\}\\
I = \{{ 0 \leq i \leq |ciudades| \yLuego res = \sum\limits_{k=0}^{i-1}ciudades[k].habitantes}\}\\
Fv = {|ciudades| - i}$
}

\vspace{0.5cm}

A) Pc \implica I \vspace{0.2cm}\\
Como el invariante es una conjunción, puedo separarlo y evaluar cada una de sus partes asumiendo como verdadero el antecedente:

\[
0 \leq i \leq |ciudades| \equiv 0 \leq 0 \leq |ciudades| 
\vspace{0.2cm}
\]
\[
res =\sum\limits_{k=0}^{i-1}ciudades[k].habitantes \equiv
 \vspace{0.2cm}\\
\]
\[
res = \{\sum\limits_{k=0}^{0-1}ciudades[k].habitantes\} = 0 
\vspace{0.5cm}\\
\]
\[
Por \ lo \ tanto, Pc \implica I 
\vspace{0.5cm}\\
\]

B)$\{I \land B \} S \{I\}$ \vspace{0.2cm}\\

Dado que queremos demostrar que la tripla de Hoare es correcta, esto pasa si y solo si vemos que $(I \land B) \implica wp(S,I)$\\
\[wp(S,I)\\\]
\[wp(res := res + ciudades[i].habitantes, i := i+1, I)\\\]
\[wp(res := res + ciudades[i].habitantes, wp(i := i+1, I))\\ \\\]
\[wp (i := i+1, I) \equiv def(i+1) \yLuego 0 \leq i+1 \leq |ciudades| \yLuego res = \sum\limits_{i=0}^{(i+1)-1}ciudades[j].habitantes\\\]
\[\equiv def(i) \land def(1) \yLuego 0\leq i+1 \leq |ciudades| \yLuego res =\sum\limits_{i=0}^{(i+1)-1}ciudades[j].habitantes  \]

\[\equiv True\land True \yLuego (0 \leq i+1 \leq |ciudades| \yLuego res = \sum\limits_{j=0}^{(i+1)-1}ciudades[j].habitantes)\\ \]
\hspace{1cm}Finalmente
\[ (0 \leq i+1 \leq |ciudades| \yLuego res = \sum\limits_{j=0}^{(i+1)-1}ciudades[j].habitantes)\\ \]
\\

\[wp(res := res + ciudades[i].habitantes, wp(i := i+1, I)) \equiv \\\]
\[wp(res := res + ciudades[i].habitantes, 0 \leq i +1\leq |ciudades| \yLuego res = {\sum\limits_{j=0}^{i}ciudades[j].habitantes}) \equiv \]
\[def(res+ciudades[i].habitantes) \yLuego 0 \leq i+1 \leq |ciudades| \yLuego res +ciudades[i].habitantes = {\sum\limits_{j=0}^{i}ciudades[j].habitantes}\]
\[def(res) \land def(ciudades[i].habitantes) \yLuego 0 \leq i+1 \leq |ciudades| \yLuego res+ciudades[i].habitantes = {\sum\limits_{j=0}^{i}ciudades[j].habitantes}\]
\[True \land 0 \leq i \leq |ciudades| \yLuego 0 \leq i+1 \leq |ciudades| \yLuego res = ({\sum\limits_{j=0}^{i}ciudades[j].habitantes})-ciudades[i].habitantes\]

\hspace{1cm}$Obs:$\\ 
\[0 \leq i \leq |ciudades| \land 0 \leq i+1 \leq |ciudades|\\  \]
\[0 \leq i \leq |ciudades| \land -1 \leq i \leq |ciudades|-1\\\]
\[0 \leq i < |ciudades|\\ \]

\hspace{1cm} Finalmente:
\[wp(S,I)=(0 \leq i < |ciudades| \yLuego res = {\sum\limits_{j=0}^{i-1}ciudades[j].habitantes})\\\]
	
\hspace{1cm} En Conclusión:\\
\[I \land B \implica wp(ciclo,I)\]
\[I \land B \implica 0 \leq i \leq |ciudades| \land i<|ciudades| \equiv 0 \leq i < |ciudades|\]
Implicado en la wp\\
\[res = \sum\limits_{j=0}^{i-1}ciudades[j].habitantes\]
Está identico en I, tanto como en la Wp
\vspace{0.5cm}\\


C) $I \land \neg B \implica Qc$\\
\[I \land \neg B \equiv 0 \leq i \leq |ciudades| \yLuego res ={\sum\limits_{j=0}^{i-1}ciudades[j].habitantes} \land i \geq |ciudades|\]
\hspace{1cm}Entonces, sabemos que $i=|ciudades|$ \ y usando esto en el otro termino de I...
\[res = {\sum\limits_{j=0}^{i-1}ciudades[j].habitantes} = {\sum\limits_{j=0}^{|ciudades|-1}ciudades[j].habitantes}\\ \\\]
\[Por \ lo \ tanto, \ I\land \neg B \implica Qc\\ \vspace{0.5cm}\\\]

Función variante propuesta: Fv = $|Ciudades|-i$ \\

D)$\{I \land B \land Fv = V_0\} S \{Fv < V_0\}$ Para demostrar esto, debemos probar que: \\
\[\{I \land B \land Fv = V_0\} \implica Wp (S, Fv < V_0 )\}\]
\\
\[Wp(S,Fv < V_0 ) \equiv Wp(res:=res+ciudades[i].habitantes, i:=i+1,|ciudades|-i<V_0)\\ \]
\[Wp(res:= res+ciudades[i].habitantes, Wp(i:=i+1,|ciudades|-i< V_0))\\\] \\
\[\hspace{1cm}Wp(i:=i+1,|ciudades|-i<V_0))  \]
\[\hspace{1cm}\equiv def (i+1) \yLuego |ciudades|-(i+1) < V_0 )\]
\[\hspace{1cm}\equiv def(i) \land def(1) \yLuego |ciudades|-i-1 < V_0 \]
\[\hspace{1cm}\equiv true \land True \yLuego |ciudades|-i-1 < V_0  \]
\hspace{1cm}Finalmente:
\[\equiv |ciudades| -i-1 < V_0\]

\[wp(res:= res+ciudades[i].habitantes, |ciudades|-i-1<V_0) \]
\[\equiv def(ciudades[i].habitantes \yLuego |ciudades|-i-1<V_0 \]
\[\equiv 0\leq i<|ciudades| \yLuego |ciudades|-i-1<V_0 \]
\hspace{1cm}Entonces:\\  \[ |ciudades|-i-1<V_0\]
\\
\[(I \land B\land V_0 = Fv) \implica wp(ciclo,Fv<V_0)\]
\hspace{1cm}Basta con tomar  $V_0 =Fv$  para  la  demostración  \\
\[|ciudades|-i-1<|ciudades-i \Iff -1<0\]


\hspace{0.5cm}E) $I \land Fv \leq 0 \implica \neg B$\\
\[I \land Fv \leq 0 \equiv (0 \leq i \leq |ciudades| \land |ciudades|-i \leq 0 \land {res = \sum\limits_{j=0}^{i-1}ciudades[j].habitantes}) \\ \]
\hspace{1cm}Entonces:
\[ (0\leq i \leq |ciudades| \land |ciudades|\leq i) \Iff (i=|ciudades|) \implica \neg(i<|ciudades|) \equiv \neg B\] \\

\[El \ ciclo\ termina ,\ y\ es\ correcto\ \]\\
\\
Para demostrar que la implementación es correcta con respecto a la especificación, debo probar por monotonía que:\\ \\
1)$ P\implicaLuego Pc$\\ \hspace{0.5cm}
\[P \implica wp(res:=0, wp(i:=0, Pc))\]
-wp(i:=0, Pc)
\[def(0) \yLuego  res = 0 \land 0 = 0 \land (\existe[unalinea]i\ent {0\leq i<|ciudades| \yLuego ciudades[i].habitantes > 50,000} \land \]  \[ \paraTodo[unalinea]i{\ent}{0 \leq i<|conexion|  \implicaLuego  (ciudades[i].habitantes \geq 0) } \land \] \[
    \paraTodo[unalinea]i{\ent}{\paraTodo[unalinea]j{\ent}{0 \leq i  < j<ciudades[i].nombre \neq ciudades[j].nombre}})\}  \equiv \]
\[true \yLuego  res = 0 \land true \land (\existe[unalinea]i\ent {0\leq i<|ciudades| \yLuego ciudades[i].habitantes > 50,000} \land \]  \[ \paraTodo[unalinea]i{\ent}{0 \leq i<|conexion|  \implicaLuego  (ciudades[i].habitantes \geq 0) } \land \] \[
    \paraTodo[unalinea]i{\ent}{\paraTodo[unalinea]j{\ent}{0 \leq i  < j<ciudades[i].nombre \neq ciudades[j].nombre}})\}  \]
Simplifico y obtengo:
\[  res = 0  \land (\existe[unalinea]i\ent {0\leq i<|ciudades| \yLuego ciudades[i].habitantes > 50,000} \land \]  \[ \paraTodo[unalinea]i{\ent}{0 \leq i<|conexion|  \implicaLuego  (ciudades[i].habitantes \geq 0) } \land \] \[
    \paraTodo[unalinea]i{\ent}{\paraTodo[unalinea]j{\ent}{0 \leq i  < j<ciudades[i].nombre \neq ciudades[j].nombre}})\}  \]
-wp(res:=0, wp(i:0, Pc))
\[def(0) \yLuego  0 = 0 \land 0 = 0 \land (\existe[unalinea]i\ent {0\leq i<|ciudades| \yLuego ciudades[i].habitantes > 50,000} \land \]  \[ \paraTodo[unalinea]i{\ent}{0 \leq i<|conexion|  \implicaLuego  (ciudades[i].habitantes \geq 0) } \land \] \[
    \paraTodo[unalinea]i{\ent}{\paraTodo[unalinea]j{\ent}{0 \leq i  < j<ciudades[i].nombre \neq ciudades[j].nombre}})\}  \equiv \]
\[true \yLuego  true  \land (\existe[unalinea]i\ent {0\leq i<|ciudades| \yLuego ciudades[i].habitantes > 50,000} \land \]  \[ \paraTodo[unalinea]i{\ent}{0 \leq i<|conexion|  \implicaLuego  (ciudades[i].habitantes \geq 0) } \land \] \[
    \paraTodo[unalinea]i{\ent}{\paraTodo[unalinea]j{\ent}{0 \leq i  < j<ciudades[i].nombre \neq ciudades[j].nombre}})\}  \]
Simplifico y finalmente obtengo:
\[(\existe[unalinea]i\ent {0\leq i<|ciudades| \yLuego ciudades[i].habitantes > 50,000} \land \]  \[ \paraTodo[unalinea]i{\ent}{0 \leq i<|conexion|  \implicaLuego  (ciudades[i].habitantes \geq 0) } \land \] \[
    \paraTodo[unalinea]i{\ent}{\paraTodo[unalinea]j{\ent}{0 \leq i  < j<ciudades[i].nombre \neq ciudades[j].nombre}})\}  \]
Que como se puede notar a simple vista, es exactamente igual a P, por lo que $P \implica Pc$  \\

2)$Pc \implicaLuego wp(while..., Qc)$ \\ \hspace{0.5cm}
Fue demostrado anteriormente con los teoremas del invariante y de terminación \\ \\
3)$ Qc \implicaLuego Q$\\ \hspace{0.5cm}
Puesto que no hay codigo luego del ciclo,y la Qc se obtiene de "patear hacia arriba" la Q, puedo afirmar Qc \implica Q, ya que la postcondición del ciclo debe cumplir la postcondición del programa.

\[Con \ esto\ queda\ demostrada\ la\ correctitud\ del\ programa\]
\vspace{2cm}\\
2. Demostrar que el valor devuelto es mayor a 50.000.
\vspace{0.3cm}\\
En el punto anterior, demostramos que, dada la implementación, es posible cumplir con la especificación propuesta. Sin embargo, esta no garantiza explícitamente que el valor de res sea mayor a 50,000. Entonces, ¿cómo podemos demostrarlo? Podemos afirmarlo gracias a la precondición, que establece que al menos una de las ciudades tiene más de 50,000 habitantes, y que además, el resto de las ciudades tienen habitantes con un valor mayor o igual a 0.\\ \\
Asimismo, dado que el tipo de dato de ciudades es IN, las cantidades de habitantes no se modificarán durante la ejecución del programa, y este solo se limitará a sumar dichos valores a res, tal como lo establece la postcondición.\\ \\
Por lo tanto, proponemos modificar la postcondición Q para asegurar que el valor de res sea mayor a 50,000. Esta modificación en Q conlleva, en consecuencia, un ajuste en nuestro invariante y en la Qc, que ahora deben reflejar la existencia de al menos una ciudad con más de 50,000 habitantes ($ciudades[j].habitantes > 50000$). De esta manera, podemos garantizar que la implementación cumple con la nueva especificación requerida y que, al finalizar el ciclo, el valor de res será mayor a 50,000. \\ \\



\[Pc = \{ res = 0 \land i = 0 \land (\existe[unalinea]i\ent {0\leq i<|ciudades| \yLuego ciudades[i].habitantes > 50,000} \land \\ \paraTodo[unalinea]i{\ent}{0 \leq i<|conexion| \] \[ \implicaLuego (ciudades[i].habitantes \geq 0) } \land
   \paraTodo[unalinea]i{\ent}{\paraTodo[unalinea]j{\ent}{0 \leq i < j<ciudades[i].nombre \neq ciudades[j].nombre}})\}   \] \\
$B = \{{i< |ciudades|}\} $ \\
$Fv = {|ciudades| - i}$ \\
\[I = \{{ 0 \leq i \leq |ciudades| \yLuego res = \sum\limits_{j=0}^{i-1}ciudades[j].habitantes \land \existe[unalinea]j{\ent}{0\leq j <|ciudades| \yLuego ciudades[j].habitantes > 50,000}}\}\]\\
Qc = $\{{res = \sum\limits_{i=0}^{|ciudades|-1}ciudades[i].habitantes \land res>50000}\}$\\ \\ \\

A) Pc \implica I \\
Como el invariante es una conjunción, puedo separarlo y evaluar cada una de sus partes asumiendo como verdadero el antecedente: \\
\[0\leq i \leq |ciudades| \equiv 0\leq 0 \leq |ciudades|\]
\[res = \sum\limits_{j=0}^{i-1}ciudades[j].habitantes \equiv 0 \equiv res = \sum\limits_{j=0}^{0-1}ciudades[j].habitantes = 0 \]
\[Luego,\]
\[\existe [unalinea]i{\ent}{0 \leq i < |ciudades| \yLuego ciudades[i].habitantes >5000} = \]
\[\existe [unalinea]j{\ent}{0 \leq j < |ciudades| \yLuego ciudades[j].habitantes >5000}  \]
\[por \ lo \ tanto, \ PC \implica I \] \\


B)$\{I \land B \} S \{I\}$ \vspace{0.2cm}: Quedó demostrado por Monotonía en el punto anterior. \\
Este nuevo componente existencial dentro del invariante garantiza que, dentro del conjunto de las ciudades, exista al menos una ciudad cuyo número de habitantes sea mayor que cero. Al aplicar esta modificación, observamos que, al realizar la implicación en los pasos subsiguientes, este nuevo componente no afecta el resultado final del cálculo del WP.
\[wp(S,I)=(0 \leq i \leq |ciudades| \yLuego res = \sum\limits_{j=0}^{i-1}ciudades[j].habitantes \ \yLuego \]
 \[\existe [unalinea]i{\ent}{0 \leq i < |ciudades| \yLuego ciudades[i].habitantes >5000} )\\\]


C) $I \land \neg B \implica Qc$\\
	\[I \land \neg B \equiv 0\leq i \leq |ciudades| \ \land \ res = \sum\limits_{j=0}^{i-1}ciudades[j].habitantes \  \land \  \existe[unalinea]j{\ent}{0\leq j <|ciudades| \yLuego ciudades[i].habitantes > 50,000} \]
	\[ \land \ i \geq |ciudades| \implica res = \sum\limits_{i=0}^{|ciudades|-1}ciudades[i].habitantes \land res>50000  \]
Luego, vale \[i \leq |ciudades| \land i \geq |ciudades| \equiv i=|ciudades|\]	
Reemplazo en Qc y obtengo lo que buscabamos:
	\[res= \sum\limits_{j=0}^{|ciudades|-1}ciudades[j].habitantes  \]
Por I sabemos que:
\[ \ \existe[unalinea]j{\ent}{0\leq j <|ciudades| \yLuego ciudades[j].habitantes > 50,000}  \]
Ahora tomo el valor de res y rompo la sumatoria para evaluar su comportamiento y poder afirmar que su valor es mayor a 50000:
	\[res = ciudades[0].habitantes+...+ciudades[j].habitantes+...\]
	\[+ciudades[|ciudades|-1]  \]
Y como 
 \[ciudades[0].habitantes \land ciudades[|ciudades|-1] .habitantes \geq 0\] \[(una \ ciudad  \   no\ puede \ tener \ habitantes\  negativos, \ eso \ nos\ pide\ la\ Pc)\]
Puedo implicar:
	\[res >50000\]

	\[Y \ como \ res= \sum\limits_{j=0}^{|ciudades|-1}ciudades[j].habitantes \ \land \ res>50000, queda \ demostrado \ que \ I \land \neg B \implica Qc   \] \\

D)$\{I \land B \land Fv = V_0\} S \{Fv < V_0\} \Iff \{I \land B \land Fv = V_0\} \implica Wp (S, Fv < V_0 )\}$\\ La modificación del invariante no cambia la demostración con respecto al punto anterior, puesto a que el componente existencial no guarda relación con la terminación del ciclo. \\

E) $I \land Fv \leq 0 \implica \neg B$\\ La modificación del invariante no cambia la demostración con respecto al punto anterior, puesto a que el componente existencial no guarda relación con la negación de la guarda.
\\
\\
Para demostrar que la implementación es correcta con respecto a la especificación, debo probar por monotonía que:\\ \\
1)$ P\implicaLuego Pc$\\ \hspace{0.5cm}
Obs: No se vé afectado por las nuevas condiciones agregadas, por lo que la demostración previa sigue valiendo.
\\ \\
2)$Pc \implicaLuego wp(while..., Qc)$ \\ \hspace{0.5cm}
Fue demostrado anteriormente con los teoremas del invariante y de terminación \\ \\
3) $Qc \implicaLuego Q$ \\
Puesto que no hay codigo luego del ciclo,y la Qc se obtiene de "patear hacia arriba" la Q, puedo afirmar Qc \implica Q, ya que la postcondición del ciclo debe cumplir la postcondición del programa.


\[Con\ esto\ queda\ demostrada\ la\ correctitud\ del\ programa\ y\ que\ res\ va\ a\ ser\ mayor\ a\ 50000.\]


\end{document}